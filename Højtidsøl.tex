%indsæt navnet på sketchen i {}
\begin{Sketch}{Højtidsøl}
\tid{0:60}

\begin{Persongalleri}
\item Gertrud (C1) \sketchrolle
\item Gertrud (c2) \sketchrolle{}
\end{Persongalleri}

\begin{Rekvisitter}
\item To ølflasker uden mærkat (R)
\end{Rekvisitter}

\fuldscene{}

\begin{Regi}
C og G kommer på forscenen, hver med en øl i hånden, som de drikker af.
\end{Regi}
\fuldscene{}
\begin{Replik}[C]
Ej, okay, jeg begynder altså at blive lidt omtåget af den her. Jeg troede da ikke der var særligt meget i? Sagde du ikke, at det var hvidtøl?
\end{Replik}

\begin{Replik}[G]
\regi{Fniser lidt.} Nej... Jeg sagde \emph{weed}-øl!
\end{Replik}
\begin{Regi}
G fniser fortsat, C stirrer på høm med åben mund. Så begynder de begge at grine fjoget, skåler og går ud.
\end{Regi}

\end{Sketch}


%Her skrives hvem der har skrevet sketchen. Bemærk \& for et 'og'-tegn 
\begin{Footer}
\TKprefix{19}\CERM
\end{Footer}
%%% Local Variables: 
%%% mode: latex
%%% TeX-master: t
%%% TeX-master: "skab2"
%%% End: 
